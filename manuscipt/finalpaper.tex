\documentclass[12pt]{article}

%% preamble: Keep it clean; only include those you need
\usepackage{amsmath}
\usepackage[margin = 1in]{geometry}
\usepackage{graphicx}
\usepackage{booktabs}
\usepackage{natbib}

% highlighting hyper links
\usepackage[colorlinks=true, citecolor=blue]{hyperref}


%% meta data

\title{Has Taylor Swift Succumbed? An Analysis of the Repetitive Nature of Pop Music and Its Influence on Major Musical Artists}
\author{Emma Graebner\\
  Department of Statistics\\
  University of Connecticut
}

\begin{document}
\maketitle

\begin{abstract}
The world of Pop music is dominated by few artists. One of those being the Grammy award winning country, pop, and indie folk singer Taylor Swift. Swift's influence on mainstream Pop music has been record-breaking. This article will analyze one song chosen at random from each of her ten studio albums (not including her re-released "(Taylor's Version")) for its proportion of unique words to total words. The data suggests that her Pop albums were the most repetitive, while her Indie Folk and Country albums were the least repetitive. The argument will be made that the songs that Top the Billboard 100 charts are more likely to be repetitive, and Swift has reflected that ideal in her own writing. This article will also analyze related artist's - Lady Gaga, Sam Smith, and Dua Lipa - popular works to compare with Swift's.   
\end{abstract}


\section{Introduction}
\label{sec:intro}

Use this section to answer three questions:
Why is the topic important/interesting?

Are Taylor Swift’s albums the same lyrical complexity? As Taylor Swift has grown as an artist, have her albums increased in proportion of non repeated words per song? What does this say about mainstream music in relation to popular artists? How do other popular artists’ works compare to her lyrical prose? 

	Because Evermore is her “most unique” album (lyrically), what does that say about the style of music she expected to be popular during the time of conception, i.e. during the Covid pandemic of 2020? The pandemic may have played a role in what she produced for her audience. Taylor Swift' demographic is teen girls who "have grown up alongside her and her music, meaning that while no longer tween-aged, their long-term fandom is embedded with nostalgia of their own girlhoods. In addition, Swift constantly evokes these feelings to court adult fans eager to return to their younger years.” \citep{rossman2022taylor}. 
	
	Other albums that came out the same year as Evermore include Chromatica by Lady Gaga and Future Nostalgia by Dua Lipa. I will analyze, at random, one song from each album to determine the uniqueness of those songs, lyrically. Comparing the complexity of those two songs (“Cool” from Future Nostalgia, and “1000 Doves” from Chromatica), I will determine how Taylor Swift’s prose is different from Lady Gaga’s and Dua Lipa’s. Similarly to Swift, Lady Gaga and Dua Lipa contributed to their respective albums by writing on every song. 
	
	Taylor Swift is the most popular artist in the country right now. Her most recent album, Midnights secured every place in the Billboard Top 10 the week of November 5th, 2022, and to this day, her single, Anti-Hero is still at #1. To compare with another song of similar relevance on the charts today, I will analyze the proportion of unique words to total words of Sam Smith's song, Unholy. This will test how her song, Labyrinth, relates to Unholy by its lyrical prose. 


  The topic of mainstream pop music is one of contention and intrigue. Stardom is the state or status of being a famous or exceptionally talented performer in the world of entertainment (cite Oxford dictionary at some point). The conversation of an artist's ability to produce lasting and iconic materials begins with the history of an artist themselves. Taylor Swift's influence on pop music is astounding. In 2013, after her album Red was released, "she became the first artist since The Beatles to spend
  six or more weeks at number one with three consecutive albums" \citep{newkey2014taylor}.
  
What has been done on this topic in the literature?

My contribution to the work that has already been done is a method of determining how unique a song is by the proportion of unrepeated lyrics in the entire song. I will do this by putting all of the text in a spreadsheet, analyzing the text for words that do not repeat in the song, and counting them. The proportion of unrepeated words to total words is what I will be comparing amongst albums.
I will also analyze the genre change of Swift's albums: from Country to Pop to Indie Folk, then again Pop. This genre change in combination with 


% roadmap
The rest of the paper is organized as follows.
The data will be presented in Section~\ref{sec:data}.
The methods are described in Section~\ref{sec:meth}.
The results are reported in Section~\ref{sec:resu}.
A discussion concludes in Section~\ref{sec:disc}.


\section{Data}
\label{sec:data}

The data that I am using to analyze the songs I have chosen at random is the individual work's lyrics. The software program Excel allowed me to organize my data in a readable format. It also provided a randomizer - the command "=randbetween(range)" - to chose each song from each album. I first found each song's lyrics on the acclaimed website azlyrics.com, formatted the words to be in one column, then pasted them into excel. Using the find and replace function, I got rid of any spaces. 

From there, I sorted the lyrics starting from A all the way to Z. This way, I was able to see all the repetitions of lyrics in one place. I then went through and manually searched for words that did not repeat, and assigned them a value of 1 in the column next to the word. After that was complete, I selected the column and it reported the sum (the number of ones I had put for each unique word). Then, I compared this number to the total number of words in the song; that is how the proportion of non repeated words was calculated.



\begin{table}[tbp]
  \caption{This table comprises the proportion of unique words in each Taylor Swift song from 10 different albums.}
  \label{tab:rv}
\centering
\begin{tabular}{rrr}
  \toprule
Album & Song & Proportion Unique Words \\ 
  \midrule
Taylor Swift & Cold As You & .2204 \\ 
  Fearless & White Horse & .2175 \\ 
  Speak Now & Sparks Fly & .2017 \\ 
  Red & I Almost Do & .1413 \\ 
  1989 & Welcome to New York & .1391 \\ 
  Reputation & ...Ready For It & .1396 \\ 
  Lover & London Boy & .1397 \\ 
  Folklore & Peace & .2334 \\ 
  Evermore & Champagne Problems & .4090 \\ 
  Midnights & Labyrinth & .1254 \\ 
   \bottomrule
\end{tabular}
\end{table}




This data explains the trend in popular mainstream music making. As you can see from the table below, The highest proportion of unrepeated words was from Taylor Swift's album entitled Evermore. The data from Champagne Problems, the song chosen at random to represent this album, has a proportion of 0.4090; almost 41 percent of the song is unique in its prose. Her most repetitive songs include Labyrinth (from Midnights) at a proportion of .1254, Welcome To New York (from 1989) with .1391, and ...Ready For It (from Reputation) having .1396 proportion. All three of these albums are considered pop music in some way. Her most unique album, Evermore, is not classified as pop, and instead is remarked as alternative and indie. 

This distinction demonstrates the tendency of pop to be more repetitive, and less novel. According to a research article entitled, "The power of repetition: repetitive lyrics in a song increase processing fluency and drive market success", a study shows that, "more repetitive songs lyrically are processed more fluently and thus adopted more broadly and quickly in the marketplace" \citep{Nunes2015power} 



\begin{table}[tbp]
  \caption{This table demonstrates the comparison of proportion unique words to its musical artist, genre, and time spent on the Billboard Top 100.}
  \label{tab:rv}
\centering
\begin{tabular}{rrrr}
  \toprule
Artist & Song & Proportion Unique Words & Genre \\ 
  \midrule
Dua Lipa & Cool & .0827 & Electronic Pop \\ 
  Lady Gaga & 1000 Doves  & .1764 & Electronic Pop \\ 
  Sam Smith & Unholy & .2017 & Dance Pop \\ 
  The Weeknd & Blinding Lights & .1760 & Electronic \\ 
     \bottomrule
\end{tabular}
\end{table}




\section{Methods}
\label{sec:meth}

Use this section to present the methodologies that will generate results by
analyzing the data.

Equation~\eqref{eq:area} is interesting. 



\section{Results}
\label{sec:resu}

Table~\ref{tab:rv} summarizes some example draws from some distributions.


Figure~\ref{fig:cars} shows the distance against the speed from this dataset.


\begin{figure}[htbp]
  \centering
  \includegraphics[width=\textwidth]{graph.png}
  \caption{This is my first figure.}
  \label{fig:graph}
\end{figure}

\section{Discussion}
\label{sec:disc}

What are the main contributions again?

What are the limitations of this study?

What are worth pursuing further in the future?



\bibliography{refs}
\bibliographystyle{mcap}

\end{document}