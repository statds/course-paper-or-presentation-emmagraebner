\documentclass[12pt]{article}

%% preamble: Keep it clean; only include those you need
\usepackage{amsmath}
\usepackage[margin = 1in]{geometry}
\usepackage{graphicx}
\usepackage{booktabs}
\usepackage{natbib}

% for space filling

% highlighting hyper links
\usepackage[colorlinks=true, citecolor=blue]{hyperref}


%% meta data

\title{Proposal: Does Time Influence the Uniqueness of Music?: A Statistical Analysis and Model Based on Taylor Swift's Discography}
\author{Emma Graebner\\
  Department of Statistics\\
  University of Connecticut
}

\begin{document}
\maketitle


\paragraph{Introduction}
My overarching research question is to discover what makes music unique using a model that takes two factors, lyrics and time since debut album was released. Chuan makes an interesting point in this article: "Music pieces or songs created by the same artist may not always sound the same, although they may sound more similar to each other than the other artists' work" \cite{chuan2013multimodal}. Taylor Swift is one of the most famous Pop singers in the modern era of music. Her music is influential, she is one of the highest grossing American artists; still selling out stadiums and earning awards for her albums and singles \citep{fogarty2021you} . Her style has changed from country to pop to indie folk as her career has advanced. My project will research the relationship between the time Taylor Swift has been an artist, to the complexity of her pieces. \citep{sloan2021taylor} 
Another reference I will use to discuss the contextualization of song lyrics in a piece is \citep{mayer2008rhyme}. 


\citep{perone2017words}

\paragraph{Specific Aims}
My hypothesis for my statistical data is: Taylor Swift’s albums are equally as lyrically complex. This is an important concept because modern pop music is increasingly repetitive. An article from ScienceDirect states that pop music's popularity depends on its repetitiveness. It claims that, through analysis, as repetitiveness increases, the chance it appears in the Billboard Top 100's list increases as well \citep{nunes2015power}. 



\paragraph{Data}
My data includes the lyrics to 9 of Taylor Swift’s songs from all nine of her albums. From “Evermore”: “closure”, from “Red”: ”The Last Time”, and from “Fearless”: “Breathe”, from “Folklore”: “Epiphany”, from “Lover”: “You Need To Calm Down”, from “Reputation”: “Call It What You Want”, from “Taylor Swift”: “Tim McGraw”, from “Speak Now”: “Mine”, and from “1989”: “Welcome To New York”. Each song has over 200 lyrics. The source of my data set is Spotify and AZlyrics.com. I will also be using SongSim to explain the repetitiveness of the songs using an analytical tool.

\paragraph{Research Design and Methods}
For my project I would like to develop a statistical model that predicts the uniqueness of a song based on two factors: the frequency of unrepeated lyrics and the year the album was released in relation to the artist’s debut album date. I will then create data files of the lyrics to the nine randomly chosen songs and use the sort and “=COUNTIF(range,criteria)” in Excel to determine the frequency of the individual words within the songs. Using that data I will create various representations including a dot plot of the linear regression of proportion unique lyrics against album year and . This regression analysis will help determine whether time and proportion of uniqueness are related. 

\paragraph{Discussion}
I expect to find that as the artist, Taylor Swift, has created more music, she has decreased the proportion of repeated words, i.e. her songs have become more complex. The existing research says that Taylor Swift is an influential and extraordinary songwriter, so this may corroborate those claims. If the investigation is not what I expect, I will choose another songwriter, ex. Lana Del Rey, or John Mayor, who have similar writing styles to determine whether time influences the complexity of an artist’s music. 

\paragraph{Conclusion}
I plan to research and analyze the relationship between proportion of unique lyrics in songs and the time since the artist's debut album was released. I will be using one song from each of Taylor Swift's albums: Taylor Swift, Fearless, Speak Now, Red, 1989, Reputation, Lover, Folklore, and Evermore as my statistical dataset. 

\bibliographystyle{chicago}
\bibliography{refs}


\end{document}