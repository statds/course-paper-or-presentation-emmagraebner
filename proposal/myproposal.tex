\documentclass[12pt]{article}

%% preamble: Keep it clean; only include those you need
\usepackage{amsmath}
\usepackage[margin = 1in]{geometry}
\usepackage{graphicx}
\usepackage{booktabs}
\usepackage{natbib}

% for space filling

% highlighting hyper links
\usepackage[colorlinks=true, citecolor=blue]{hyperref}


%% meta data

\title{Proposal: Does Time Influence the Uniqueness of Music?: A Statistical Analysis and Model Based on Taylor Swift's Discography}
\author{Emma Graebner\\
  Department of Statistics\\
  University of Connecticut
}

\begin{document}
\maketitle


\paragraph{Introduction}
My overarching research question is to discover what makes music unique using a model that takes two factors, lyrics and time Taylor Swift is one of the most famous Pop singers in the modern era of music. Her music is influential, she is one of the highest grossing American artists; still selling out stadiums and earning awards for her albums and singles \citep{fogarty2021you} . Her style has changed from country to pop to indie folk as her career has advanced. My project will research the relationship between the time Taylor Swift has been an artist, to the complexity of her pieces. \citep{sloan2021taylor} 


\citep{perone2017words}

\paragraph{Specific Aims}


\paragraph{Data}
My data includes the lyrics to 9 of Taylor Swift’s songs from all nine of her albums. From “Evermore”: “closure”, from “Red”: ”The Last Time”, and from “Fearless”: “Breathe”, from “Folklore”: “Epiphany”, from “Lover”: “You Need To Calm Down”, from “Reputation”: “Call It What You Want”, from “Taylor Swift”: “Tim McGraw”, from “Speak Now”: “Mine”, and from “1989”: “Welcome To New York”. Each song has over 200 lyrics. The source of my data set is Spotify and AZlyrics.com.  
\paragraph{Research Design and Methods}


\paragraph{Discussion}
I expect to find that as the artist, Taylor Swift, has created more music, she has decreased the proportion of repeated words, i.e. her songs have become more complex. The existing research says that Taylor Swift is an influential and extraordinary songwriter, so this may corroborate those claims. If the investigation is not what I expect, I will choose another songwriter, ex. Lana Del Rey, or John Mayor, who have similar writing styles to determine whether time influences the complexity of an artist’s music. 

\paragraph{Conclusion}


\bibliographystyle{chicago}
\bibliography{refs}


\end{document}